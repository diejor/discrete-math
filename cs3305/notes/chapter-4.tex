\chapter{Number Theory and Cryptography}

\section{Divisibility and Modular Arithmetic}

\begin{definition}
  Let $a$ and $b$ be integers with $a \neq 0$. We say that $a$ \textbf{divides} $b$ if there exists an integer $c$ such that $b = ac$. We write $a \mid b$ to denote that $a$ divides $b$.
\end{definition}

\begin{theorem}
    Let $a$, $b$, and $c$ be integers. 
    \begin{enumerate}
        \item If $a \mid b$ and $a \mid c$, then $a \mid (b + c)$.
        \item If $a \mid b$, then $a \mid (bc)$ for all integers $c$.
        \item If $a \mid b$ and $b \mid c$, then $a \mid c$.
    \end{enumerate}
\end{theorem}

\begin{corollary}
    Let $a$, $b$, and $c$ be integers, where $a \neq 0$. If $a \mid b$ and $a \mid c$, then $a \mid (bx + cy)$ for all integers $x$ and $y$.
\end{corollary}

\begin{definition}
    \textbf{The Division Algorithm.} Let $a$ and $d$ a positive integers. Then there exist unique integers $q$ and $r$ such that $a = dq + r$ and $0 \leq r < d$.
\end{definition}

\begin{definition}
    In the equality given in the division algorithm, $d$ is called the \textbf{divisor}, $a$ is called the \textbf{dividend}, $q$ is called the \textbf{quotient}, and $r$ is called the \textbf{remainder}. This notation is used to express the quotient and remainder: 
    \begin{equation*}
        q = \frac{a}{d} \quad \text{and} \quad r = a \bmod d
    \end{equation*}
\end{definition}


\subsection{exercises}

\begin{exercise}[22]
    Let $m$ be a positive integer. Show that $a \mod m = b \mod m$ if and only if \cmod{a}{b}{m}.
\end{exercise}
