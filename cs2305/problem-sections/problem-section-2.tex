
% TEMPLATE FROM https://github.com/jubnoske08/linear_algebra

\documentclass{extarticle}

\usepackage[12pt]{extsizes}
\usepackage{times}
\usepackage[shortlabels]{enumitem}
\usepackage{microtype} 
\usepackage{mathtools}
\usepackage{commath}




\usepackage{tcolorbox}
\tcbuselibrary{theorems, breakable, skins}

% Problem environment
\newtcbtheorem{prob}% environment name
              {Problem}% Title text
  {enhanced, % tcolorbox styles
  attach boxed title to top left={xshift = 4mm, yshift=-2mm},
  colback=blue!5, colframe=black, colbacktitle=blue!3, coltitle=black,
  boxed title style={size=small,colframe=gray},
  fonttitle=\bfseries,
  separator sign none
  }%
  {} 
\newenvironment{problem}[1]{\begin{prob*}{#1}{}}{\end{prob*}}

% Definition environment
\newtcbtheorem{defn}% environment name
			  {Definition}% Title text
  {enhanced, % tcolorbox styles
  attach boxed title to top left={xshift = 4mm, yshift=-2mm},
  colback=green!5, colframe=black, colbacktitle=green!3, coltitle=black,
  boxed title style={size=small,colframe=gray},
  fonttitle=\bfseries,
  separator sign none
  }%
  {}
\newenvironment{definition}[1]{\begin{defn*}{#1}{}}{\end{defn*}}


\title{Problem Section 2}
\author{Diego Rodrigues}
\date{September 20, 2023}

\begin{document}
\maketitle

\begin{definition}{Big-O Notation}
	Let $f$ and $g$ be functions from the set of integers or the set of real numbers to the set of real numbers. 
	We say that $f(x)$ is $O(g(x))$ if there are constants $C$ and $k$ such that
	$$\abs{f(x)} \le C\abs{g(x)} \text{ whenever } x > k$$
\end{definition}
Also, $C$ and $k$ are referred as the witnesses of the Big-O notation, and are those possible non unique pair of constants that satisfy the definition above.
This also applies to $\Omega$ and $\Theta$ notations.

\begin{definition}{Big-$\Omega$ Notation}
	Let $f$ and $g$ be functions from the set of integers or the set of real numbers to the set of real numbers. 
	We say that $f(x)$ is $\Omega(g(x))$ if there are constants $C$ and $k$ such that
	$$\abs{f(x)} \ge C\abs{g(x)} \text{ whenever } x > k$$
\end{definition}

\begin{definition}{Big-$\Theta$ Notation}
	Let $f$ and $g$ be functions from the set of integers or the set of real numbers to the set of real numbers.
	We say that $f(x)$ is $\Theta(g(x))$ if there are constants $C_1, C_2$ and $k$ such that
	$$C_1\abs{g(x)} \le \abs{f(x)} \le C_2\abs{g(x)} \text{ whenever } x > k$$
\end{definition}

\begin{problem}{1}
    Describe the shaded area in terms of sets A, B, C.
\end{problem}

\begin{problem}{2}
    For sets $A$, $B$, and $C$, is $(A - B) - C -> A - (B - C)$?
\end{problem}

\begin{problem}{3}
    in a group of 30 children, 10 like apples, 10 like oranges, 10 like bananas
    and 6 like all three (nobody likes just two). How many children do not like any of
    the three fruits? 
\end{problem}
Apply inclusion-exclusion principle 

\begin{problem}{4}
    Give examples of functions from integers to integers that (a) one-to-one but not onto,
    (b) onto but not one-to-one, (c) one-to-one and onto.
\end{problem}
a) $f(x) = 2x$ 
b) $f(x) = \lfloor{x/2}\rfloor$
c) $f(x) = x$

\begin{problem}{5}
    If possible, give exaples of functions from A to A where A 
\end{problem}

\begin{problem}{6}
    Define "countable set"; show that ${\frac{a}{2}, \frac{a}{4}, ...}$ where
    $a$ is in ${1, 2, 3, 4, 5, 6, 7, 8, 9}$ is countable.
\end{problem}

\begin{problem}{7}
    Consider set of all strings over alphabet ${0, 1, 2, ..., 9}$, prove or 
    disprove claim that set is countable.
\end{problem}
\end{document}