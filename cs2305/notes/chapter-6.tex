\chapter{Counting}

\section{The Basics of Counting}

\newcommand{\productrule}{
    \begin{definition}
        If a procedure can be broken into 2 different tasks, and for each there are $n_1$ and $n_2$ ways to do it respectively, then there are $n_1n_2$ ways to do the procedure.
    \end{definition}
}

\productrule

\subsection{Problem Section (book)}

% exercise 1
\begin{exercise}
    Just use of the two rules.
    \begin{enumerate}
        \item use product rule       
        \item use sum rule
    \end{enumerate}
\end{exercise}


% exercise 2
\begin{exercise}
    For each 27 floor there are 37 offices. If you were to assign a person in a office there would be $27 \times 37$ ways to do it.
\end{exercise}

\begin{exercise}
    There are 10 problems and 4 choices for each problem.
    \begin{enumerate}
        \item the student has 4 choices for each 10 question, then there are $4^{10}$ ways to do it.    
        \item the student has 5 choices for each 10 question, then there are $5^10$ ways to do it.
    \end{enumerate}
\end{exercise}

% exercise 4
\begin{exercise}
    Use extended product rule, there would be 12 colors, 2 female or male and 3 sizes. So there would be $12 \times 2 \times 3$ types of shirts.
\end{exercise}

% exercise 5
\begin{exercise}
    There would be $6 \times 7$ pairs of airlines.
\end{exercise}

\begin{exercise}
    Same as 5. $4 \times 5$ auto routes.
\end{exercise}

% exercise 7
\begin{exercise}
    By extended product rule there would be $26 \times 26 \times 26$ ways to do it or $26^3$.
\end{exercise}

% exercise 8
\begin{exercise}
    Each time one letter is chosen it substract 1 to the following choices to account for no repetition, then we would have $26 \times (26 - 1) \times (24 - 1 - 1)$ ways three consecutive non-repeated letters can be constructed.
\end{exercise}

% exercise 9
\begin{exercise}
    The same as 8 but with two letters, there would be $25 \times (25 - 1)$ ways to do it. \textbf{ASK PROF} it should be 676 and I get 600.
\end{exercise}

% exercise 10
\begin{exercise}
    There are $2^8$ bit strings of length 8.
\end{exercise}

% exercise 11
\begin{exercise}
    There are $2^8$ bit strings of length 10 that ends with both begin and end with a 1.
\end{exercise}

% exercise 12
\begin{exercise}
    There are $2^6 + 2^5 + 2^4 + 2^3 + 2^2 + 2 = 2^7 -1$ bit strings of length 6 or less.
\end{exercise}

% exercise 13
\begin{exercise}
    There are $n$ bit strings of 1s not exeding n. Not counting the empty string.
\end{exercise}

% exercise 14
\begin{exercise}
    There are $2^{n-2}$ bit strings of length n that begin and end with 1. Not counting the empty string.
\end{exercise}

% exercise 15
\begin{exercise}
    There are $26^4 + 26^3 + 26^2 + 26^1$ strings with lowercase letters of length 4 or less (not counting the empty string).
\end{exercise}

% exercise 16
\begin{exercise}
    There are $4*26^3 + 3*26^2 + 2*26^1 + 1$ strings. \textbf{ASK PROF}.
\end{exercise}


