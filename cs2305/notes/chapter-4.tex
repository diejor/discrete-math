\chapter{Divisibility and Modular Arithmetic}

\newcommand{\divisorDef}{
    \begin{definition}
        Let $a$ and $b$ be integers with $a \neq 0$. We say that $a$ \textbf{divides} $b$ if there exists an integer $c$ such that $b = ac$ (or equivalently, if $\frac{b}{a}$ is an integer). IWhen $a$ divides $b$ we say that $a$ is a \textbf{factor} or \textbf{divisor} of $b$, and that $b$ is a \textbf{multiple} of $a$. The notation $a \mid b$ denotes that $a$ divides $b$. We write $a \nmid b$ to denote that $a$ does not divide $b$.
    \end{definition}
}

\newcommand{\divisibilityThm}{
    \begin{theorem}
        Let $a$, $b$, and $c$ be integers. Where $a \neq 0$. Then
        \begin{enumerate}[(j)]
            \item if $a \mid b$ and $a \mid c$, then $a \mid (b + c)$.
            \item if $a \mid b$, then $a \mid (bc)$ for all integers $c$.
            \item if $a \mid b$ and $b \mid c$, then $a \mid c$.
        \end{enumerate}
    \end{theorem}
}

\newcommand{\compoundDivisibility}{
    \begin{corollary}
        If $a$, $b$, and $c$ are integers, where $a \neq 0$, such that $a \mid b$ and $a \mid c$, then $a \mid (mb + nc)$ whenever $m$ and $n$ are integers. 
    \end{corollary}
}

\newcommand{\divisionAlgThm}{
    \begin{theorem}[Division Algorithm]
        Let $a$ be an integer and $d$ a positive integer. Then there are unique integers $q$ and $r$, with $0 \leq r < d$, such that $a = dq + r$.
    \end{theorem}
}

\newcommand{\divisorDividentDef}{
    \begin{definition}
        From the division algorithm, d is the \textit{divisor} and a is the \textit{dividend}, q is called the \textit{quotient} and r is called the \textit{remainder}. This notation is used to express the quotient and remainder:
        \begin{equation*}
           q = a\ \mathbf{div}\ d,   \quad  r = a\ \mathbf{mod}\ d
        \end{equation*}
    \end{definition}
}

\newcommand{\modulusDef}{
    \begin{definition}
        If $a$ and $b$ are integers and $m$ is a positive integer, then $a$ is $congruent\ to\ b\ module m$ if $m$ divides $a - b$. We write $a \equiv b\mod m$ to denote that $a$ is congruent to $b$ modulo $m$. We say that $a \equiv b \mod m$ is a \textbf{congruence} and $m$ is its \textbf{modulus} (plural \textbf{moduli}). If $a$ and $b$ are not congruent modulo $m$, we write $a \not\equiv b \mod m$.
    \end{definition}
}  

\newcommand{\modulusThm}{
    \begin{theorem}
        Let $a$ and $b$ be integers, and let $m$ be a positive integer. Then $a \equiv b \mod m$ if and only if $a \mod m = b\ \mod m$.
    \end{theorem}
}

\newcommand{\reminderModulusThm}{
    \begin{theorem}
        Let $m$ be a positive integer. The Integers $a$ and $b$ are congruent modulo $m$ if and only if there is an integer $k$ such that $a = b + km$.
    \end{theorem}
}

\newcommand{\mulAddModulusThm}{
    \begin{theorem}
        Let $m$ be a positive integer. If $a \equiv b \mod m$ and $c \equiv d \mod m$, then
    \begin{equation*}
        a + c \equiv b + d \mod m \qquad ac \equiv bd \mod m.
    \end{equation*}
    \end{theorem}
}

\newcommand{\modulusDistributCor}{
    \begin{corollary}
        Let $m$ be a positive integer and let $a$, $b$ be integers. Then
        \begin{equation*}
            (a + b)\mod{m} = ((a \mod{m}) + (b \mod{m}))\mod{m}            
        \end{equation*}
        and
        \begin{equation*}
            (ab)\mod{m} = ((a \mod{m})(b \mod{m}))\mod{m}
        \end{equation*}
    \end{corollary}
}


\divisorDef
\divisibilityThm
\compoundDivisibility
\divisionAlgThm
\divisorDividentDef
\modulusDef

\modulusThm
\reminderModulusThm
\mulAddModulusThm
\modulusDistributCor


\section{Integer Representations and Algorithms}
\resetcounters

\newcommand{\intBaseThm}{
    \begin{theorem}
        Let $b$ be an integer greater than $1$. Then if $n$ is a positive integer, it can be expressed uniquely in the form
        \begin{equation*}
            n = a_kb^k + a_{k-1}b^{k-1} + \cdots + a_1b + a_0
        \end{equation*}
        where $k$ is a nonnegative integer, $a_0, a_1, \ldots, a_k$ are nonnegative integers less than $b$, and $a_k \neq 0$.
    \end{theorem}
}

\newcommand{\baseExpansionAlg}{
    \begin{algorithm}
        \begin{algorithmic}
            \Function{base b expansion}{$n$, $b$: positive integers with $b > 1$}
                \State $q \gets n$
                \State $k \gets 0$
                \While{$q \neq 0$}
                    \State $a_k \gets q \mod{b}$
                    \State $q \gets q \div{b}$
                    \State $k \gets k + 1$
                \EndWhile
                \State \Return $(a_{k-1}a_{k-2}\cdots a_1a_0)$
            \EndFunction
        \end{algorithmic}
    \end{algorithm}
}

\newcommand{\addIntAlg}{
    \begin{algorithm}
        \begin{algorithmic}
            \Function{add}{$a$, $b$: nonnegative integers, where $a = (a_{n-1}a_{n-2}\cdots a_1a_0)_2$ and $b = (b_{n-1}b_{n-2}\cdots b_1b_0)_2$}
                \State $c \gets 0$
                \For{$j \gets 0$ to $n-1$}
                    \State $d \gets \floor{(a_i + b_i + c)/2}$
                    \State $s_i \gets (a_i + b_i + c) - 2d$
                    \State $c \gets d$
                \EndFor
                \State $s_n \gets c$
                \State \Return $(s_ns_{n-1}\cdots s_1s_0)_2$
            \EndFunction
        \end{algorithmic}
    \end{algorithm}
}

\newcommand{\mulIntAlg}{
    \begin{algorithm}
        \begin{algorithmic}
            \Function{multiply}{$a$, $b$: nonnegative integers, where $a = (a_{n-1}a_{n-2}\cdots a_1a_0)_2$ and $b = (b_{n-1}b_{n-2}\cdots b_1b_0)_2$}
                \For {$j \gets 0$ to $n-1$}
                    \If{$b_i = 1$}
                        \State $c \gets a \ll j$
                    \Else
                        \State $c \gets 0$
                    \EndIf
            \Comment $\{ c_0, c_1, \ldots, c_{n-1} \}$ are the partial products]
                    \State $p \gets 0$
                    \For{$j \gets 0$ to $n-1$}
                        \State $p \gets add(p, c_j)$
                    \EndFor
                    \State \Return $p$ 
                \EndFor
            \EndFunction
        \end{algorithmic}
    \end{algorithm}
}

\newcommand{\divmodAlg}{
    \begin{algorithm}
        \begin{algorithmic}
            \Function{divide}{$a$: integer, $d$ positive integer}
                \State $q \gets 0$
                \State $r \gets \abs{a}$
                \While {$r \ge d$}
                    \State $r \gets r - d$
                    \State $q \gets q + 1$
                \EndWhile
                \If {$a < 0$ and $r > 0$}
                    \State $r \gets d - r$
                    \State $q \gets -(q + 1)$
                \EndIf
                \Comment $\{ q, r \}$ q = a \div{d}, r = a \mod{d}
                \State \Return $(q, r)$ 
            \EndFunction
        \end{algorithmic}
    \end{algorithm}
}

\newcommand{\fastModAlg}{
    \begin{algorithm}
        \begin{algorithmic}
            \Function{modular exponentiation}{$b$: integer, $n = (a_{k-1}, a_{k-2}, \ldots, a_1, a_0)_2$, $m$: positive integer}
                \State $x \gets 1$
                \State $power \gets b \mod{m}$
                \For{$i \gets 0$ to $k-1$}
                    \If{$a_i = 1$}
                        \State $x \gets (x \cdot power) \mod{m}$
                    \EndIf
                    \State $power \gets (power \cdot power) \mod{m}$
                \EndFor
                \State \Return $x$
            \EndFunction
        \end{algorithmic}
    \end{algorithm}
}

\intBaseThm
\baseExpansionAlg
\addIntAlg
\mulIntAlg
\divmodAlg
\fastModAlg